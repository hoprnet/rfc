\section{RFC-0001: RFC Life Cycle, Process and
Structure}\label{rfc-0001-rfc-life-cycle-process-and-structure}

\begin{itemize}
\tightlist
\item
  \textbf{RFC Number:} 0001
\item
  \textbf{Title:} RFC Life Cycle, Process and Structure
\item
  \textbf{Status:} Raw
\item
  \textbf{Author(s):} Qianchen Yu (@QYuQianchen), Tino Breddin
  (@tolbrino)
\item
  \textbf{Created:} 2025-02-20
\item
  \textbf{Updated:} 2025-08-20
\item
  \textbf{Version:} v0.2.0 (Raw)
\item
  \textbf{Supersedes:} N/A
\item
  \textbf{Related Links:} none
\end{itemize}

\subsection{1. Abstract}\label{abstract}

This RFC defines the life cycle, contribution process, versioning
system, governance model, and document structure for RFCs at HOPR. It
outlines stages, naming conventions, validation rules, and formatting
standards that MUST be followed to ensure consistency and clarity across
all RFC submissions. The process ensures iterative development with
feedback loops and transparent updates with pull requests (PR).

\subsection{2. Motivation}\label{motivation}

HOPR project requires a clear and consistent process for managing
technical proposals, documenting protocol architecture. A well-defined
life cycle MUST be established to maintain coherence, ensure quality,
and streamline future development.

\subsection{3. Terminology}\label{terminology}

The key words ``MUST'', ``MUST NOT'', ``REQUIRED'', ``SHALL'', ``SHALL
NOT'', ``SHOULD'', ``SHOULD NOT'', ``RECOMMENDED'', ``MAY'', and
``OPTIONAL'' in this document are to be interpreted as described in
{[}01{]}.

\textbf{Draft:} An RFC is considered a draft from the moment it is
proposed for review. A draft MUST include a clear summary, context, and
initial technical details. Drafts MUST follow the v0.x.x versioning
scheme, with each version being independently implementable. A draft
version is assigned as soon as the first PR is created.

\subsection{4. Specification}\label{specification}

\subsubsection{4.1. RFC Life Cycle Stages}\label{rfc-life-cycle-stages}

\paragraph{\texorpdfstring{4.1.1. \textbf{Mermaid Diagram for RFC Life
Cycle
Stages}}{4.1.1. Mermaid Diagram for RFC Life Cycle Stages}}\label{mermaid-diagram-for-rfc-life-cycle-stages}

\begin{Shaded}
\begin{Highlighting}[]
\NormalTok{graph TD}
\NormalTok{    A["\textasciigrave{}**Raw**}
\NormalTok{      Initial unstructured ideas\textasciigrave{}"]}
\NormalTok{    B["\textasciigrave{}**Discussion**}
\NormalTok{      Collaborative refinement with feedback\textasciigrave{}"]}
\NormalTok{    C["\textasciigrave{}**Review**}
\NormalTok{      Focused evaluation to assess feasibility\textasciigrave{}"]}
\NormalTok{    D["\textasciigrave{}**Draft** (v0.x.x)}
\NormalTok{      Structured proposal ready for development, each version independently implementable\textasciigrave{}"]}
\NormalTok{    E["\textasciigrave{}**Implementation**}
\NormalTok{      PR Merge\textasciigrave{}"]}
\NormalTok{    F["\textasciigrave{}**Finalized** (v1.0.0)}
\NormalTok{      Stable and complete RFC\textasciigrave{}"]}
\NormalTok{    G["\textasciigrave{}**Errata** (v1.0.x)}
\NormalTok{      Minor technical corrections post{-}finalization\textasciigrave{}"]}
\NormalTok{    H["\textasciigrave{}**Superseded**}
\NormalTok{      New RFC required for significant updates\textasciigrave{}"]}
\NormalTok{    I["\textasciigrave{}**Rejected**}
\NormalTok{      Documented reasons for rejection\textasciigrave{}"]}

\NormalTok{    A {-}{-}\textgreater{} B}
\NormalTok{    B {-}{-}\textgreater{} C}
\NormalTok{    C {-}{-}\textgreater{} D}
\NormalTok{    D {-}{-}\textgreater{} E}
\NormalTok{    E {-}{-}\textgreater{} F}
\NormalTok{    F {-}{-}\textgreater{} G}
\NormalTok{    F {-}{-}\textgreater{} H}
\NormalTok{    A {-}{-}\textgreater{} I}
\end{Highlighting}
\end{Shaded}

\paragraph{\texorpdfstring{4.1.2. \textbf{Stage
Descriptions:}}{4.1.2. Stage Descriptions:}}\label{stage-descriptions}

\begin{itemize}
\tightlist
\item
  \textbf{Raw:} The RFC \textbf{MUST} begin as a raw draft reflecting
  initial ideas. The draft MAY contain incomplete details but MUST
  provide a clear objective.
\item
  \textbf{Discussion:} Upon submission of the initial PR, the RFC number
  and \texttt{v0.1.0} version are assigned. Feedback SHALL be gathered
  via PRs, with iterative updates reflected in version increments
  \texttt{(v0.x.x)}.
\item
  \textbf{Review:} The RFC \textbf{MUST} undergo at least one review
  cycle. The draft \textbf{SHOULD} incorporate significant feedback and
  each iteration \textbf{MUST} be independently implementable.
\item
  \textbf{Draft:} The RFC moves into active development and refinement.
  Each update \textbf{SHALL} increment the version (\texttt{v0.x.x}) to
  indicate progress.
\item
  \textbf{Implementation:} Merging to the main branch signifies
  readiness for practical use, triggering the finalization process.
\item
  \textbf{Finalized:} The RFC is considered stable and complete, with
  version \texttt{v1.0.0} assigned. Only errata modifications are
  permitted afterward.
\item
  \textbf{Errata:} Minor technical corrections post-finalization
  \textbf{MUST} be documented and result in a patch version increment
  (\texttt{v1.0.x}). Errata are technical corrections or factual updates
  made after an RFC has been finalized. They \textbf{MUST NOT} alter the
  intended functionality or introduce new features.
\item
  \textbf{Superseded:} Significant updates requiring functionality
  changes \textbf{MUST} be documented in a new RFC, starting at
  \texttt{v2.0.0} or higher. The original RFC must include information
  that it has been superseded, accompanied with a link to the new RFC
  that supersedes it.
\item
  \textbf{Rejected:} If an RFC does not progress past the discussion
  stage, reasons \textbf{MUST} be documented.
\end{itemize}

\subsubsection{4.2. File Structure}\label{file-structure}

\begin{Shaded}
\begin{Highlighting}[]
\NormalTok{RFC{-}0001{-}rfc{-}life{-}cycle{-}process/}
\NormalTok{│}
\NormalTok{├── 0001{-}rfc{-}life{-}cycle{-}process.md}
\NormalTok{├── errata/}
\NormalTok{│   └── 0001{-}v1.0.1{-}erratum.md}
\NormalTok{└── assets/}
\NormalTok{    └── life{-}cycle{-}overview.png}
\end{Highlighting}
\end{Shaded}

\begin{center}\rule{0.5\linewidth}{0.5pt}\end{center}

\subsubsection{4.3. Validation Rules}\label{validation-rules}

\begin{itemize}
\tightlist
\item
  Directory \textbf{MUST} be prefixed with uppercased ``RFC'', followed
  by its RFC number, and a succinct title all in lowercase joined by
  hyphens. E.g. \texttt{RFC-0001-rfc-life-cycle-process}
\item
  Main file \textbf{MUST} be prefixed with its RFC number and a succinct
  title all in lowercase joined by hyphens. E.g.
  \texttt{0001-rfc-life-cycle-process.md}
\item
  All assets \textbf{MUST} reside in the \texttt{assets/} folder.
\item
  Errata \textbf{MUST} reside in the \texttt{errata/} folder.
\end{itemize}

\subsubsection{4.4. RFC Document
Structure}\label{rfc-document-structure}

All RFCs \textbf{MUST} follow a consistent document structure to ensure
readability and maintainability.

\paragraph{4.4.1. Metadata Preface}\label{metadata-preface}

Every RFC \textbf{MUST} begin with the following metadata structure:

\begin{Shaded}
\begin{Highlighting}[]
\FunctionTok{\# RFC{-}XXXX: [Title]}

\SpecialStringTok{{-} }\NormalTok{**RFC Number:** XXXX}
\SpecialStringTok{{-} }\NormalTok{**Title:** }\CommentTok{[}\OtherTok{Title in Title Case}\CommentTok{]}
\SpecialStringTok{{-} }\NormalTok{**Status:** Raw | Discussion | Review | Draft | Implementation | Finalized | Errata | Rejected | Superseded}
\SpecialStringTok{{-} }\NormalTok{**Author(s):** }\CommentTok{[}\OtherTok{Name (GitHub Handle)}\CommentTok{]}
\SpecialStringTok{{-} }\NormalTok{**Created:** YYYY{-}MM{-}DD}
\SpecialStringTok{{-} }\NormalTok{**Updated:** YYYY{-}MM{-}DD}
\SpecialStringTok{{-} }\NormalTok{**Version:** vX.X.X (Status)}
\SpecialStringTok{{-} }\NormalTok{**Supersedes:** RFC{-}YYYY (if applicable) | N/A}
\SpecialStringTok{{-} }\NormalTok{**Related Links:** }\CommentTok{[}\OtherTok{RFC{-}XXXX}\CommentTok{](../RFC{-}XXXX{-}[slug]/XXXX{-}[slug].md)}\NormalTok{ | none}
\end{Highlighting}
\end{Shaded}

\paragraph{4.4.2. Reference Styles}\label{reference-styles}

RFCs \textbf{MUST} use two distinct reference styles:

\subparagraph{4.4.2.1. RFC-to-RFC
References}\label{rfc-to-rfc-references}

\begin{itemize}
\tightlist
\item
  RFC references to other HOPR RFCs \textbf{MUST} be listed in the
  metadata's \textbf{Related Links:} field
\item
  Format:
  \texttt{{[}RFC-XXXX{]}(../RFC-XXXX-{[}slug{]}/XXXX-{[}slug{]}.md)}
\item
  Multiple references \textbf{SHALL} be separated by commas
\item
  If no RFC references exist, the field \textbf{MUST} contain ``none''
\item
  Example:
  \texttt{{[}RFC-0002{]}(../RFC-0002-mixnet-keywords/0002-mixnet-keywords.md),\ {[}RFC-0004{]}(../RFC-0004-hopr-packet-protocol/0004-hopr-packet-protocol.md)}
\end{itemize}

\subparagraph{4.4.2.2. External References}\label{external-references}

\begin{itemize}
\item
  External references \textbf{MUST} be listed in a dedicated
  \texttt{\#\#\ References} section at the end of the document
\item
  References \textbf{MUST} use sequential numbering with zero-padding:
  {[}01{]}, {[}02{]}, etc.
\item
  In-text citations \textbf{MUST} use the numbered format: ``as
  described in {[}01{]}''
\item
  Reference format \textbf{SHOULD} follow academic citation style:

\begin{verbatim}
[XX] Author(s). (Year). [Title](URL). _Publication_, Volume(Issue), pages.
\end{verbatim}
\item
  Example:

\begin{verbatim}
[01] Chaum, D. (1981). [Untraceable Electronic Mail, Return Addresses, and Digital Pseudonyms](https://www.freehaven.net/anonbib/cache/chaum-mix.pdf). _Communications of the ACM, 24_(2), 84-90.
\end{verbatim}
\end{itemize}

\paragraph{4.4.3. Required Sections}\label{required-sections}

All RFCs \textbf{MUST} include the following sections:

\begin{enumerate}
\def\labelenumi{\arabic{enumi}.}
\tightlist
\item
  \textbf{Metadata Preface} (as defined in 4.4.1)
\item
  \textbf{Abstract} - Brief summary of the RFC's purpose and scope
\item
  \textbf{References} - External citations (if any)
\end{enumerate}

\subsection{5. Design Considerations}\label{design-considerations}

\begin{itemize}
\tightlist
\item
  Modular RFCs \textbf{SHOULD} be preferred.
\item
  PR system \textbf{MUST} be the primary mechanism for contribution,
  review, and errata handling.
\end{itemize}

\subsection{6. Compatibility}\label{compatibility}

\begin{itemize}
\tightlist
\item
  New RFCs \textbf{MUST} maintain backward compatibility unless
  explicitly stated.
\item
  Errata \textbf{MUST NOT} introduce backward-incompatible changes.
\item
  Breaking changes \textbf{MUST} be reflected in a major version
  increment (\texttt{v2.0.0}).
\end{itemize}

\subsection{7. Security Considerations}\label{security-considerations}

\begin{itemize}
\tightlist
\item
  Security review phase \textbf{MUST} be included before finalization.
\item
  Errata \textbf{MUST} undergo security review if impacting critical
  components.
\end{itemize}

\subsection{8. Drawbacks}\label{drawbacks}

\begin{itemize}
\tightlist
\item
  Strict naming conventions \textbf{MAY} limit creative flexibility.
\end{itemize}

\subsection{9. Alternatives}\label{alternatives}

\begin{itemize}
\tightlist
\item
  Collaborative document editing tools, e.g.~hackmd.
\end{itemize}

\subsection{10. Unresolved Questions}\label{unresolved-questions}

\begin{itemize}
\tightlist
\item
  Handling emergency RFCs
\item
  Enforcing cross-RFC dependencies
\item
  Formal approval timeline for errata
\end{itemize}

\subsection{11. Future Work}\label{future-work}

\begin{itemize}
\tightlist
\item
  Automated validation tools
\item
  CI/CD integration for automated versioning and errata checks
\item
  Web interface for publishing RFCs
\end{itemize}

\subsection{12. References}\label{references}

{[}01{]} Bradner, S. (1997).
\href{https://datatracker.ietf.org/doc/html/rfc2119}{Key words for use
in RFCs to Indicate Requirement Levels}. \emph{IETF RFC 2119}.

{[}02{]} \href{https://www.rfc-editor.org/styleguide/}{RFC Editor Style
Guide}. RFC Editor.

{[}03{]} \href{https://github.com/rust-lang/rfcs}{Rust RFC Process}.
Rust Language Team.

{[}04{]} \href{https://rfc.zeromq.org}{ZeroMQ RFC Process}. ZeroMQ
Community.

{[}05{]} \href{https://github.com/vacp2p/rfc-index}{VACP2P RFC Index}.
Vac Research.
