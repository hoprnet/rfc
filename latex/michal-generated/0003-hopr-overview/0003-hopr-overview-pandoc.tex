\rfcnumber{0003}
\rfctitle{HOPR Overview}
\rfcdate{October 2025}
\rfcauthor{Tino Breddin (@tolbrino)}
\section{RFC-0003: HOPR Overview}\label{rfc-0003-hopr-overview}

\begin{itemize}
\tightlist
\item
  \textbf{RFC Number:} 0003
\item
  \textbf{Title:} HOPR Overview
\item
  \textbf{Status:} Finalised
\item
  \textbf{Author(s):} Tino Breddin (@tolbrino)
\item
  \textbf{Created:} 2025-09-11
\item
  \textbf{Updated:} 2025-10-27
\item
  \textbf{Version:} v1.0.0 (Finalised)
\item
  \textbf{Supersedes:} none
\item
  \textbf{Related Links:}
  \href{../RFC-0002-mixnet-keywords/0002-mixnet-keywords.md}{RFC-0002},
  \href{../RFC-0004-hopr-packet-protocol/0004-hopr-packet-protocol.md}{RFC-0004},
  \href{../RFC-0005-proof-of-relay/0005-proof-of-relay.md}{RFC-0005},
  \href{../RFC-0007-economic-reward-system/0007-economic-reward-system.md}{RFC-0007},
  \href{../RFC-0008-session-protocol/0008-session-protocol.md}{RFC-0008},
  \href{../RFC-0009-session-start-protocol/0009-session-start-protocol.md}{RFC-0009},
  \href{../RFC-0010-automatic-path-discovery/0010-automatic-path-discovery.md}{RFC-0010},
  \href{../RFC-0011-application-protocol/0011-application-protocol.md}{RFC-0011}
\end{itemize}

\subsection{1. Abstract}\label{abstract}

This RFC provides an introductory overview of the HOPR network (also
referred to as HOPRnet) and its associated protocol stack. HOPR is a
decentralised and incentivised mix network that enables
privacy-preserving communication by routing messages through multiple
relay nodes using onion routing.

HOPR's key innovation is the proof-of-relay mechanism, which addresses
the challenge of establishing economically sustainable anonymous
communication networks. By combining cryptographic proofs with economic
incentives, HOPR enables scalable privacy infrastructure that becomes
stronger with increased adoption, in contrast to volunteer-based
networks that struggle with sustainability and performance issues.

This document serves as the primary entry point for understanding the
HOPR network as outlined in these RFCs. It introduces the network
architecture and protocol stack at a conceptual level and provides
references to additional RFCs that define specific implementation
details. The intended audience includes researchers, developers, and
infrastructure operators seeking to understand or implement
privacy-preserving communication systems based on the HOPR protocol.

\subsection{2. Motivation}\label{motivation}

In the contemporary digital environment, privacy-preserving
communication has become essential for safeguarding user data,
supporting freedom of expression, and maintaining confidentiality in
both personal and professional contexts. Conventional internet protocols
provide inadequate protection for privacy, as metadata and traffic
patterns can be analysed to infer sensitive information about users and
their communications.

The HOPR protocol addresses these privacy challenges by implementing a
decentralised mix network that:

\begin{itemize}
\tightlist
\item
  \textbf{Provides metadata privacy}: Unlike traditional communication
  networks that expose communication patterns, HOPR obscures
  sender-receiver relationships through traffic mixing and onion routing
  {[}01, 02{]}
\item
  \textbf{Offers economic incentives}: Node operators are compensated
  for relaying traffic, thereby creating an economically sustainable
  privacy infrastructure
\item
  \textbf{Ensures decentralisation}: No single entity controls the
  network, mitigating the risks of censorship and eliminating single
  points of failure
\item
  \textbf{Maintains accessibility}: Applications can integrate HOPR's
  privacy capabilities without requiring users to understand complex
  cryptographic concepts
\end{itemize}

The HOPR protocol is designed to be transport-agnostic, enabling
operation over standard internet infrastructures while preserving robust
privacy guarantees. By combining established cryptographic primitives
with novel incentive mechanisms, HOPR offers a practical and scalable
solution for privacy-preserving communication.

\subsection{3. Terminology}\label{terminology}

All terminology used in this document, including general mix network
concepts and HOPR-specific definitions, is provided in
\href{../RFC-0002-mixnet-keywords/0002-mixnet-keywords.md}{RFC-0002}.
That document serves as the authoritative reference for the terminology
and conventions adopted across the HOPR RFC series.

\subsection{4. Network Overview}\label{network-overview}

The HOPR network is a decentralised, peer-to-peer mix network that
provides privacy-preserving communication through multi-hop routing. The
network architecture consists of several key components that work
together to ensure metadata privacy whilst incentivising participation
through economic rewards.

\subsubsection{4.1 Network Architecture}\label{network-architecture}

The HOPR network comprises different node roles based on their function
in message routing:

\begin{itemize}
\tightlist
\item
  \textbf{Entry nodes}: nodes that initiate communication sessions and
  inject messages into the network
\item
  \textbf{Relay nodes}: intermediate nodes that forward messages along
  routing paths and receive payment for their relay services
\item
  \textbf{Exit nodes}: final relay nodes in a path that deliver messages
  to their intended destinations\\
\item
  \textbf{Payment infrastructure}: on-chain payment channels that enable
  efficient microtransactions between nodes without requiring a
  blockchain transaction for each payment
\end{itemize}

Every HOPR node can simultaneously act as an entry node, relay node, and
exit node depending on the context of different message flows. The
distinction between these roles is functional rather than structural,
being dependent on a node's position within a specific routing path.

\subsubsection{4.2 Path Construction}\label{path-construction}

Messages in the HOPR network are routed through multi-hop paths to
provide privacy protection. Path construction involves three phases:

\begin{enumerate}
\def\labelenumi{\arabic{enumi}.}
\tightlist
\item
  \textbf{Path discovery}: nodes discover available relay nodes through
  automated probing mechanisms detailed in
  \href{../RFC-0010-automatic-path-discovery/0010-automatic-path-discovery.md}{RFC-0010}.
  This process identifies which nodes are reachable, reliable, and have
  open payment channels.
\item
  \textbf{Path selection}: senders choose routing paths based on
  multiple criteria, including privacy requirements, expected latency,
  relay costs, and node reliability. The selection algorithm balances
  these trade-offs according to application needs.
\item
  \textbf{Onion routing}: messages are encrypted in multiple layers
  using the Sphinx packet format {[}02, 03{]}, with each relay node able
  to decrypt only one layer to reveal the next hop whilst keeping the
  sender, final destination, and full path hidden.
\end{enumerate}

\subsubsection{4.3 Economic Incentives}\label{economic-incentives}

The HOPR network employs economic incentives to ensure sustainable
operation and encourage node participation:

\begin{itemize}
\tightlist
\item
  \textbf{Micropayments}: relay nodes receive small probabilistic
  payments for each message they forward. Payments are made through
  tickets that have a winning probability, enabling efficient
  micropayments without excessive on-chain transactions.
\item
  \textbf{Proof of relay}: cryptographic proofs ensure that relay nodes
  actually forward messages before receiving payment. This mechanism is
  detailed in
  \href{../RFC-0005-proof-of-relay/0005-proof-of-relay.md}{RFC-0005} and
  prevents nodes from claiming payment without providing service.
\item
  \textbf{Payment channels}: unidirectional payment channels between
  nodes enable efficient microtransactions without high blockchain fees
  {[}04{]}. Channels are established on-chain but allow many off-chain
  payments, settling only periodically or when channels close.
\item
  \textbf{Staking rewards}: nodes that stake tokens and maintain open
  payment channels receive additional rewards as described in
  \href{../RFC-0007-economic-reward-system/0007-economic-reward-system.md}{RFC-0007},
  creating incentives for network participation beyond per-message
  payments.
\end{itemize}

\subsubsection{4.4 Privacy Properties}\label{privacy-properties}

The network architecture provides several key privacy guarantees through
its layered security approach:

\begin{itemize}
\tightlist
\item
  \textbf{Sender anonymity}: relay nodes cannot determine the original
  sender of a message due to onion routing. Each node only knows the
  immediate previous hop, not the ultimate source {[}05{]}.
\item
  \textbf{Receiver anonymity}: intermediate nodes cannot identify the
  final recipient of a message. Only the exit node knows the final
  destination, but not the original sender {[}05{]}.
\item
  \textbf{Unlinkability}: observers cannot link multiple messages from
  the same sender or to the same receiver {[}05{]}. Different messages
  may take different paths, and the encryption prevents correlation.
\item
  \textbf{Traffic analysis resistance}: random delays introduced by the
  mixer component
  (\href{../RFC-0006-hopr-mixer/0006-hopr-mixer.md}{RFC-0006}) and
  packet mixing prevent timing correlation attacks {[}06{]}. This
  ensures that an observer cannot correlate incoming and outgoing
  packets based on timing patterns.
\end{itemize}

These properties hold even against an adversary who controls a subset of
the network nodes, as long as at least one honest node exists in each
routing path.

\subsection{5. Protocol Overview}\label{protocol-overview}

The HOPR protocol stack consists of multiple layers that work together
to provide privacy-preserving communication with economic incentives.
This section provides a high-level overview of the protocol components
and their interactions.

\subsubsection{5.1 Protocol Architecture}\label{protocol-architecture}

The HOPR protocol is organised into five layers, arranged as follows:

\begin{codebubbleenv}
┌─────────────────────────────────────┐
│        Application Layer            │
├─────────────────────────────────────┤
│      Session Management Layer       │
├─────────────────────────────────────┤
│       HOPR Application Protocol     │
├─────────────────────────────────────┤
│        HOPR Packet Protocol         │
├─────────────────────────────────────┤
│        Transport Layer              │
└─────────────────────────────────────┘
\end{codebubbleenv}

From top to bottom, these layers provide the following functionalities:

\begin{itemize}
\tightlist
\item
\textbf{Application layer}: Support for applications and services\\
\textbf{Session management layer}: Session establishment and data
transfer\\
\textbf{HOPR application protocol}: Message routing and
protocol multiplexing\\
\textbf{HOPR packet protocol}: Onion routing and
encryption\\
\textbf{Transport layer}: Network communication
\end{itemize}

\subsubsection{5.2 Core Protocol
Components}\label{core-protocol-components}

\paragraph{5.2.1 HOPR Packet Protocol}\label{hopr-packet-protocol}

The HOPR packet protocol
(\href{../RFC-0004-hopr-packet-protocol/0004-hopr-packet-protocol.md}{RFC-0004})
defines the fundamental packet format and processing rules that enable
onion routing:

\begin{itemize}
\tightlist
\item
  \textbf{Onion encryption}: multi-layer encryption ensures that each
  relay node can decrypt only one layer to reveal the next hop's
  address, maintaining sender and destination anonymity throughout the
  routing process.
\item
  \textbf{Sphinx-based design}: based on the Sphinx packet format
  {[}03{]} with extensions for incentivisation. Sphinx provides compact
  headers and strong cryptographic guarantees about packet
  unlinkability.
\item
  \textbf{Fixed packet size}: all packets have identical size (including
  header, payload, and proof-of-relay information) to prevent traffic
  analysis based on packet size {[}06{]}. To achieve this,
  variable-length messages are padded to the maximum size.
\item
  \textbf{Single-use reply blocks (SURBs)}: SURBs enable recipients to
  send reply messages back to anonymous senders without knowing their
  identity, supporting bidirectional communication whilst preserving
  anonymity.
\end{itemize}

\paragraph{5.2.2 Proof of Relay}\label{proof-of-relay}

The proof of relay mechanism
(\href{../RFC-0005-proof-of-relay/0005-proof-of-relay.md}{RFC-0005})
ensures that relay nodes actually forward packets before receiving
payment:

\begin{itemize}
\tightlist
\item
  \textbf{Cryptographic proofs}: each packet contains cryptographic
  challenges that can only be solved by a node that successfully
  delivers a packet to the next hop. The solution serves as mathematical
  proof that the relay service was performed.
\item
  \textbf{Payment integration}: proofs are cryptographically bound to
  payment tickets. Relay nodes can only claim payment by presenting
  valid proofs, ensuring that compensation is tied to actual work
  performed.
\item
  \textbf{Fraud prevention}: the mechanism detects and prevents nodes
  from claiming payment without providing relay services. Invalid proofs
  are rejected, and repeated fraud attempts can result in channel
  closure and stake slashing.
\end{itemize}

\paragraph{5.2.3 Traffic Mixing}\label{traffic-mixing}

The HOPR mixer
(\href{../RFC-0006-hopr-mixer/0006-hopr-mixer.md}{RFC-0006}) provides
traffic analysis resistance through temporal mixing:

\begin{itemize}
\tightlist
\item
  \textbf{Temporal mixing}: introduces random delays to packets before
  forwarding, breaking timing correlations between incoming and outgoing
  packets {[}01, 06{]}. This prevents attackers from linking packets
  based on timing patterns.
\item
  \textbf{Configurable delays}: supports configurable minimum delay and
  delay range parameters, allowing nodes to balance privacy protection
  against latency requirements based on their threat model and
  application needs.
\item
  \textbf{Per-packet randomisation}: each packet receives an
  independently generated random delay, ensuring that timing patterns
  cannot be exploited even when observing multiple packets.
\end{itemize}

The mixer operates as a priority queue ordered by release timestamps,
efficiently managing packets even under high-load conditions.

\paragraph{5.2.4 Session Management}\label{session-management}

Session protocols provide higher-level communication primitives on top
of the basic packet transport:

\begin{itemize}
\tightlist
\item
  \textbf{Session establishment}:
  \href{../RFC-0009-session-start-protocol/0009-session-start-protocol.md}{RFC-0009}
  defines how nodes establish communication sessions with capability
  negotiation, session identifier exchange, and keep-alive mechanisms.
\item
  \textbf{Data transfer}:
  \href{../RFC-0008-session-protocol/0008-session-protocol.md}{RFC-0008}
  provides both reliable and unreliable data transmission modes.
  Reliable mode includes acknowledgements, retransmissions, and in-order
  delivery, whilst unreliable mode offers lower latency for applications
  that can tolerate packet loss.
\item
  \textbf{Message fragmentation}: sessions handle segmentation of large
  messages into multiple packets and reassembly at the destination,
  transparently managing the fixed packet size constraint.
\item
  \textbf{Connection management}: session lifecycle management including
  error handling, timeout management, and graceful termination.
\end{itemize}

\paragraph{5.2.5 Economic System}\label{economic-system}

The economic reward system
(\href{../RFC-0007-economic-reward-system/0007-economic-reward-system.md}{RFC-0007})
incentivises network participation through multiple mechanisms:

\begin{itemize}
\tightlist
\item
  \textbf{Staking rewards}: nodes that stake tokens receive rewards
  proportional to their stake, encouraging long-term network commitment
  and providing economic security.
\item
  \textbf{Payment channels}: unidirectional payment channels enable
  efficient micropayments between nodes {[}04{]}. Channels are funded
  on-chain but support many off-chain transactions, minimising
  blockchain costs.
\item
  \textbf{Fair distribution}: rewards are distributed equitably based on
  staked amounts and network participation, ensuring that nodes with
  open channels and good connectivity receive appropriate compensation.
\item
  \textbf{Quality-of-service incentives}: the reward system considers
  node reliability and availability, incentivising operators to maintain
  high-quality service.
\end{itemize}

\subsubsection{5.3 Protocol Flow}\label{protocol-flow}

A typical message transmission through the HOPR network follows this
flow:

\begin{enumerate}
\def\labelenumi{\arabic{enumi}.}
\tightlist
\item
  \textbf{Path discovery}: the sender discovers available relay nodes
  through active probing and constructs a routing path based on network
  topology, channel availability, and performance metrics.
\item
  \textbf{Session establishment}: if reliable delivery or bidirectional
  communication is required, the sender establishes a session with the
  recipient using the session start protocol. For simple one-way
  messages, this step may be skipped.
\item
  \textbf{Packet construction}: the message (possibly fragmented into
  multiple packets) is encrypted in multiple layers using onion
  encryption. Each layer includes routing information for one hop and
  cryptographic challenges for proof of relay.
\item
  \textbf{Routing}: packets are forwarded through the selected path,
  with each relay node:

  \begin{itemize}
  \tightlist
  \item
    Removing one layer of encryption to reveal the next hop's address
  \item
    Applying random delays through the mixer component
  \item
    Solving cryptographic challenges to generate proofs of relay
  \item
    Claiming payment tickets upon successful delivery to the next hop
  \end{itemize}
\item
  \textbf{Delivery}: the exit node delivers the packet to the intended
  recipient, who can decrypt the final layer to access the message
  content.
\end{enumerate}

\subsubsection{5.4 Integration Points}\label{integration-points}

The HOPR protocol provides multiple integration points to support
various applications and use cases:

\begin{itemize}
\tightlist
\item
  \textbf{Application protocol}:
  \href{../RFC-0011-application-protocol/0011-application-protocol.md}{RFC-0011}
  defines a lightweight multiplexing layer that allows multiple
  higher-level protocols to coexist over the HOPR packet transport,
  similar to port numbers in TCP/UDP.
\item
  \textbf{Transport independence}: the protocol can operate over
  different network transports (TCP, UDP, QUIC, etc.), making it
  deployable in various network environments without requiring specific
  infrastructure.
\item
  \textbf{API compatibility}: through the session protocols, HOPR
  provides familiar networking APIs (stream-based and datagram-based) to
  ease application integration and lower the barrier to adoption.
\item
  \textbf{Extensibility}: the modular design allows for protocol
  extensions and improvements without breaking existing implementations.
  New features can be negotiated during session establishment through
  capability flags.
\end{itemize}

\subsection{6. References}\label{references}

{[}01{]} Chaum, D. (1981).
{Untraceable Electronic Mail, Return Addresses, and Digital Pseudonyms}.
\emph{Communications of the ACM, 24}(2), 84-90.
\href{https://www.freehaven.net/anonbib/cache/chaum-mix.pdf}{\underline{https://www.freehaven.net/anonbib/cache/chaum-mix.pdf}}
\\
{[}02{]} Reed, M. G., Syverson, P. F., \& Goldschlag, D. M. (1998).
{Anonymous Connections and Onion Routing}. \emph{IEEE Journal on Selected Areas in
Communications, 16}(4), 482-494.
\href{https://www.onion-router.net/Publications/JSAC-1998.pdf}{\underline{https://www.onion-router.net/Publications/JSAC-1998.pdf}}
\\
{[}03{]} Danezis, G., \& Goldberg, I. (2009).
{Sphinx: A Compact and Provably Secure Mix Format}. \emph{2009 30th IEEE Symposium
on Security and Privacy}, 262-277.
\href{https://cypherpunks.ca/~iang/pubs/Sphinx_Oakland09.pdf}{\underline{https://cypherpunks.ca/~iang/pubs/Sphinx\_Oakland09.pdf}}
\\
{[}04{]} Poon, J., \& Dryja, T. (2016).
{The Bitcoin Lightning Network: Scalable Off-Chain Instant Payments}. Lightning
Network Whitepaper.
\href{https://lightning.network/lightning-network-paper.pdf}{\underline{https://lightning.network/lightning-network-paper.pdf}}
\\
{[}05{]} Pfitzmann, A., \& Köhntopp, M. (2001).
{Anonymity, Unobservability, and Pseudonymity---A Proposal for Terminology}. In
\emph{Designing Privacy Enhancing Technologies} (pp.~1-9). Springer.
\href{https://www.freehaven.net/anonbib/cache/terminology.pdf}{\underline{https://www.freehaven.net/anonbib/cache/terminology.pdf}}
\\
{[}06{]} Raymond, J. F. (2001).
{Traffic Analysis: Protocols, Attacks, Design Issues, and Open Problems}. In
\emph{Designing Privacy Enhancing Technologies} (pp.~10-29). Springer.
\href{https://www.freehaven.net/anonbib/cache/raymond-thesis.pdf}{\underline{https://www.freehaven.net/anonbib/cache/raymond-thesis.pdf}}