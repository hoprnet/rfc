\documentclass{article}
\usepackage{fontspec}
\setmonofont{SourceCodePro-Regular}[Path=./fonts/,Extension=.ttf]
\newfontfamily\sourcecodepro{SourceCodePro-Regular}[Path=./fonts/,Extension=.ttf]
\usepackage{xcolor}
\definecolor{bubbleback}{HTML}{CECFE6}
\definecolor{rfcbody}{HTML}{2C4190}
\RequirePackage[most]{tcolorbox}
\tcbuselibrary{breakable,skins,varwidth}
\newtcbox{\codebubble}{
  enhanced,
  breakable,
  parbox=true,
  varwidth boxed,
  varwidth upper,
  tcbox raise base,
  colback=bubbleback,
  colframe=bubbleback,
  boxrule=0pt,
  arc=4pt,
  left=4pt,
  right=4pt,
  top=2pt,
  bottom=2pt,
  boxsep=0pt,
  fontupper=\sourcecodepro\small\color{rfcbody},
  before upper=\strut,
  after upper=\strut
}
\begin{document}In Großbritannien beginnt an diesem Montag der Prozess gegen eine Reihe von Autobauern, denen im Zuge des sogenannten Dieselskandals eine Manipulation der Abgaswerte ihrer Fahrzeuge vorgeworfen wird. In dem auf drei Monate angesetzten Gerichtsverfahren vor dem High Court in London soll die Frage geklärt werden, ob in Dieselautos von Mercedes-Benz, Ford, Renault, Nissan und den Stellantis-Marken Peugeot und Citroën eingebaute Systeme dazu dienten, die Abgasvorschriften zu umgehen.

Der Prozess gegen die fünf angeklagten Hersteller soll dabei auch einen Präzedenzfall für andere Hersteller schaffen und könnte den Weg für Entschädigungszahlungen in Milliardenhöhe ebnen. Die Klagen wurden im Namen von 1,6 Millionen Autofahrerinnen und Autofahrern gegen 14 Automobilhersteller eingereicht.

Hersteller weisen Vorwürfe zurück
Mercedes-Benz und der US-Autobauer Ford wiesen diese Vorwürfe als »unbegründet« zurück. Der japanische Hersteller Nissan äußerte sich bislang nicht. Stellantis, der Mutterkonzern der französischen Marken Peugeot und Citroën, sowie der französische Hersteller Renault sagten, dass die von ihnen verkauften Fahrzeuge den damaligen Vorschriften entsprochen hätten.
This is a test: \codebubble{Der Dieselskandal war im September 2015 bekannt geworden. ABC very/long_code-string.with+lots:of=breaks}
\end{document}