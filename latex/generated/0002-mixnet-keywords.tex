\section{RFC-0002: Common mixnet terms and
keywords}\label{rfc-0002-common-mixnet-terms-and-keywords}

\begin{itemize}
\tightlist
\item
  \textbf{RFC Number:} 0002
\item
  \textbf{Title:} Common mixnet terms and keywords
\item
  \textbf{Status:} Draft
\item
  \textbf{Author(s):} Tino Breddin (@tolbrino)
\item
  \textbf{Created:} 2025-08-01
\item
  \textbf{Updated:} 2025-09-04
\item
  \textbf{Version:} v0.1.0 (Draft)
\item
  \textbf{Supersedes:} none
\item
  \textbf{Related Links:} none
\end{itemize}

\subsection{1. Abstract}\label{abstract}

This RFC provides a glossary of common terms and keywords related to
mixnets and the HOPR protocol specifically. It aims to establish a
shared vocabulary for developers, researchers, and users involved in the
HOPR project.

\subsection{2. Motivation}\label{motivation}

The HOPR project involves a diverse community of people with different
backgrounds and levels of technical expertise. A shared vocabulary is
essential for clear communication and a common understanding of the
concepts and technologies used in the project. This RFC aims to provide
a single source of truth for the terminology used in the HOPR ecosystem.

\subsection{3. Terminology}\label{terminology}

\begin{itemize}
\item
  \textbf{Mixnet:} Also known as a \textbf{Mix network} is a routing
  protocol that creates hard-to-trace communications by using a chain of
  proxy servers known as mixes which take in messages from multiple
  senders, shuffle them, and send them back out in random order to the
  next destination.
\item
  \textbf{Node:} A process which implements the HOPR protocol and
  participates in the mixnet. Nodes can be run by anyone. A node can be
  a sender, destination or a relay node which helps to relay messages
  through the network. Also referred to as ``peer'' {[}01, 02{]}.
\item
  \textbf{Sender:} The node that initiates communication by sending out
  a packet through the mixnet. This is typically an application which
  wants to send a message anonymously {[}01, 02{]}.
\item
  \textbf{Destination:} The node that receives a message sent through
  the mixnet. Also referred to as ``receiver'' in some contexts {[}01,
  02{]}.
\item
  \textbf{Peer}: A node that is connected to another node in the p2p
  network. Each peer has a unique identifier and can communicate with
  other peers. The terms ``peer'' and ``node'' are often used
  interchangeably.
\item
  \textbf{Cover Traffic:} Artificial data packets introduced into the
  network to obscure traffic patterns with adaptive noise. These data
  packets can be generated on any node and are used to make it harder to
  distinguish between real user traffic and dummy traffic {[}01, 03{]}.
\item
  \textbf{Path:} The route a message takes through the mixnet, defined
  as a sequence of hops between sender and destination. A path can be
  direct from sender to destination, or it can go through multiple relay
  nodes before reaching the destination. Also referred to as ``message
  path'' {[}01, 02{]}.
\item
  \textbf{Forward Path:} A path that is used to deliver a packet only in
  the direction from the sender to the destination.
\item
  \textbf{Return Path:} A path that is used to deliver a packet in the
  opposite direction than the forward path. The return path MAY be
  disjoint with the forward path.
\item
  \textbf{Relay Node:} A node that forwards messages from one node to
  another in the mixnet. Relay nodes help to obscure the sender's
  identity by routing messages through multiple nodes {[}01, 02{]}.
\item
  \textbf{Hop:} A relay node in the message path that is neither the
  sender nor the destination. E.g. a 0-hop message is sent directly from
  the sender to the destination, while a 1-hop message goes through one
  relay node before reaching the destination. The terms ``hop'' and
  ``relay'' are often used interchangeably {[}01, 02{]}. More hops in
  the path generally increase the anonymity of the message, but also
  increase latency and cost.
\item
  \textbf{Mix Nodes:} These are the proxy servers that make up the
  mixnet. They receive messages from multiple senders, shuffle them, and
  then send them back out in a random order {[}01{]}.
\item
  \textbf{Layered Encryption:} A technique where a message is wrapped in
  successive layers of encryption. Each intermediary node (or hop) can
  only decrypt its corresponding layer, revealing the next destination
  in the path {[}01, 04{]}.
\item
  \textbf{Metadata:} Data that provides information about other data. In
  the context of mixnets, this includes things like the sender's and
  destination's IP addresses, the size of the message, and the time it
  was sent or received. Mixnets work to shuffle this metadata to protect
  user privacy {[}01, 06{]}.
\item
  \textbf{Onion Routing:} A technique for anonymous communication over a
  network. It involves encrypting messages in layers, analogous to the
  layers of an onion, which are then routed through a series of network
  nodes {[}04{]}.
\item
  \textbf{Public Key Cryptography:} A cryptographic system that uses
  pairs of keys: public keys, which may be disseminated widely, and
  private keys, which are known only to the owner. This is used to
  encrypt messages sent through the mixnet {[}01{]}.
\item
  \textbf{Sphinx:} A packet format that ensures unlinkability and
  layered encryption. It uses a fixed-size packet structure to resist
  traffic analysis {[}02{]}.
\item
  \textbf{Symmetric Encryption:} A type of encryption where the same key
  is used to both encrypt and decrypt data {[}05{]}.
\item
  \textbf{Traffic Analysis:} The process of intercepting and examining
  messages in order to deduce information from patterns in
  communication. Mixnets are designed to make traffic analysis very
  difficult {[}01{]}.
\item
  \textbf{Forward Message:} A packet that is sent along the forward
  path. Also referred to as ``forward packet''.
\item
  \textbf{Reply Message:} A packet that is sent along the return path.
  Also referred to as ``reply packet''.
\end{itemize}

\subsection{4. References}\label{references}

{[}01{]} Chaum, D. (1981).
\href{https://www.freehaven.net/anonbib/cache/chaum-mix.pdf}{Untraceable
Electronic Mail, Return Addresses, and Digital Pseudonyms}.
\emph{Communications of the ACM, 24}(2), 84-90.

{[}02{]} Danezis, G., \& Goldberg, I. (2009).
\href{https://cypherpunks.ca/~iang/pubs/Sphinx_Oakland09.pdf}{Sphinx: A
Compact and Provably Secure Mix Format}. \emph{2009 30th IEEE Symposium
on Security and Privacy}, 262-277.

{[}03{]} K. Sampigethaya and R. Poovendran, A Survey on Mix Networks and
Their Secure Applications. Proceedings of the IEEE, vol.~94, no. 12,
pp.~2142-2181, Dec.~2006.

{[}04{]} Reed, M. G., Syverson, P. F., \& Goldschlag, D. M. (1998).
\href{https://www.onion-router.net/Publications/JSAC-1998.pdf}{Anonymous
Connections and Onion Routing}. \emph{IEEE Journal on Selected Areas in
Communications, 16}(4), 482-494.

{[}05{]} Shannon, C. E. (1949). Communication Theory of Secrecy Systems.
\emph{Bell System Technical Journal, 28}(4), 656-715. DOI:
10.1002/j.1538-7305.1949.tb00928.x

{[}06{]} Cheu, A., Smith, A., Ullman, J., Zeber, D., \& Zhilyaev, M.
(2019, April). Distributed differential privacy via shuffling. In Annual
international conference on the theory and applications of cryptographic
techniques (pp.~375-403). Cham: Springer International Publishing.
