\section{RFC-0006: HOPR Mixer}\label{rfc-0006-hopr-mixer}

\begin{itemize}
\tightlist
\item
  \textbf{RFC Number:} 0006
\item
  \textbf{Title:} HOPR Mixer
\item
  \textbf{Status:} Implementation
\item
  \textbf{Author(s):} Tino Breddin (@tolbrino)
\item
  \textbf{Created:} 2025-08-14
\item
  \textbf{Updated:} 2025-09-04
\item
  \textbf{Version:} v0.1.0 (Draft)
\item
  \textbf{Supersedes:} N/A
\item
  \textbf{Related Links:}
  \href{../RFC-0002-mixnet-keywords/0002-mixnet-keywords.md}{RFC-0002},
  \href{../RFC-0004-hopr-packet-protocol/0004-hopr-packet-protocol.md}{RFC-0004}
\end{itemize}

\subsection{1. Abstract}\label{1-abstract}

This RFC describes the HOPR Mixer component, a critical element of the
HOPR mixnet that introduces temporal mixing to break timing correlation
between incoming and outgoing packets. The mixer applies random delays
to packets, effectively destroying temporal patterns that could be used
for traffic analysis. This specification details the
mixer\textquotesingle s design, implementation requirements, and
integration points to enable consistent implementations across different
HOPR nodes.

\subsection{2. Motivation}\label{2-motivation}

In mixnets, simply forwarding packets through multiple hops is
insufficient to prevent traffic analysis attacks. Adversaries can
correlate packets by observing timing patterns, even if packet contents
are encrypted and routes are obscured. Without temporal mixing, an
observer monitoring network traffic can potentially link incoming and
outgoing packets based on their timing relationships.

The HOPR Mixer addresses this attack vector by:

\begin{itemize}
\tightlist
\item
  Breaking temporal correlations between packet arrival and departure
  times
\item
  Providing configurable delay parameters to balance anonymity and
  performance
\item
  Using an efficient queuing mechanism that maintains packet ordering
  based on release times
\item
  All while supporting high-throughput scenarios without compromising
  mixing effectiveness
\end{itemize}

\subsection{3. Terminology}\label{3-terminology}

Terms defined in
\href{../RFC-0002-mixnet-keywords/0002-mixnet-keywords.md}{RFC-0002} are
used. Additional mixer-specific terms include:

\emph{mixing delay}: A random time interval added to a
packet\textquotesingle s transit time through a node to prevent timing
correlation attacks.

\emph{release timestamp}: The calculated time when a delayed packet
should be forwarded from the mixer.

\emph{mixing buffer}: A priority queue that holds packets ordered by
their release timestamps.

\subsection{4. Specification}\label{4-specification}

\subsubsection{4.1. Overview}\label{41-overview}

The HOPR Mixer follows a flow-based design which is split into these
steps:

\begin{enumerate}
\def\labelenumi{\arabic{enumi}.}
\tightlist
\item
  Accepts packets from upstream components
\item
  Assigns random delays to each packet
\item
  Stores packets in a time-ordered buffer
\item
  Releases packets when their delay expires
\end{enumerate}

\subsubsection{4.2. Configuration
Parameters}\label{42-configuration-parameters}

The mixer accepts the following configuration parameters:

\begin{enumerate}
\def\labelenumi{\arabic{enumi}.}
\tightlist
\item
  \emph{min\_delay}: Minimum delay applied to packets (default: 0ms)
\item
  \emph{delay\_range}: Range from minimum to maximum delay (default:
  200ms)
\end{enumerate}

The actual delay for each packet is randomly selected from a chosen
distribution over the interval
\texttt{{[}min\_delay,\ min\_delay\ +\ delay\_range{]}}.

\subsubsection{4.3. Core Components}\label{43-core-components}

\paragraph{4.3.1. Delay Assignment}\label{431-delay-assignment}

When a packet arrives, the mixer:

\begin{enumerate}
\def\labelenumi{\arabic{enumi}.}
\tightlist
\item
  Generates a random delay using a cryptographically secure random
  number generator
\item
  Calculates the release timestamp as
  \texttt{current\_time\ +\ random\_delay}
\item
  Wraps the packet with its release timestamp
\item
  Puts the wrapped packet into a buffer ordered by the release timestamp
\end{enumerate}

Random delay generation:

\begin{itemize}
\tightlist
\item
  MUST use a CSPRNG with sufficient entropy
\item
  MUST be independent per packet (no reuse/correlation across packets)
\item
  SHOULD allow uniform distribution as the baseline; other distributions
  MAY be added via configuration
\end{itemize}

Note: Uniform distribution is a simple baseline. More advanced
strategies like Poisson mixing (as used in Loopix {[}01{]}) can provide
stronger anonymity properties by making packet timings less
distinguishable from cover traffic patterns.

\paragraph{4.3.2. Mixing Buffer}\label{432-mixing-buffer}

The mixer maintains packets in a data structure where:

\begin{itemize}
\tightlist
\item
  Packets are ordered by their release timestamps
\item
  The packet with the earliest release time is always at the top
\item
  Insertion and extraction operations have O(log n) complexity
\item
  If multiple packets share the same \texttt{release\_time}, the
  ordering MUST be stable FIFO by insertion sequence
\end{itemize}

This ensures efficient processing even under high load conditions.

\subsubsection{4.4. Operational Behavior}\label{44-operational-behavior}

\paragraph{4.4.1. Packet Processing
Flow}\label{441-packet-processing-flow}

\begin{verbatim}
1. Packet arrives at mixer via Sender
2. Random delay is generated: delay ∈ [min_delay, min_delay + delay_range]
3. Release timestamp calculated: release_time = now() + delay
4. Packet wrapped with timestamp and inserted into buffer
5. Receiver woken if sleeping
5a. If the inserted packet has an earlier `release_time` than the current head, re-arm the timer to the new head
6. When current_time ≥ release_time, packet is released to Receiver
6a. Upon wake (including after system sleep), release all packets with `release_time` ≤ current_time before sleeping again
\end{verbatim}

\paragraph{4.4.2. Timer Management}\label{442-timer-management}

The mixer requires a timer that is able to:

\begin{itemize}
\tightlist
\item
  Wake the mixer at the next packet\textquotesingle s
  \texttt{release\_time}
\item
  Use minimal system calls and context switches
\item
  Handle concurrent access safely
\item
  Use a monotonic clock source (not wall-clock) for computing
  \texttt{release\_time}
\item
  Handle system sleep/clock adjustments by releasing all overdue packets
  immediately upon wake
\end{itemize}

NOTE: The need for a dedicated timer MAY be satisfied automatically when
using a RTOS and its native waking mechanisms.

\subsubsection{4.5. Special Cases}\label{45-special-cases}

\paragraph{4.5.1. Zero Delay
Configuration}\label{451-zero-delay-configuration}

When both \texttt{min\_delay} and \texttt{delay\_range} are zero:

\begin{itemize}
\tightlist
\item
  Packets pass through without mixing
\item
  Original packet order is preserved
\item
  Useful for testing or non-anonymous operation modes
\end{itemize}

\subsection{5. Design Considerations}\label{5-design-considerations}

\subsubsection{5.1. Performance
Optimization}\label{51-performance-optimization}

An implementation should prioritize:

\begin{itemize}
\tightlist
\item
  \textbf{Minimal allocations}: Pre-allocated buffer reduces memory
  pressure
\item
  \textbf{Efficient data structures}: Binary heap provides O(log n)
  operations
\item
  \textbf{Lock minimization}: Fine-grained locking for concurrent access
\item
  \textbf{Timer efficiency}: Single shared timer reduces system
  overhead, including minimizing runtime system overhead by using a
  single thread
\end{itemize}

\subsubsection{5.2. Abuse Resistance and Resource
Limits}\label{52-abuse-resistance-and-resource-limits}

\begin{itemize}
\tightlist
\item
  \textbf{Timing attacks}: Random delays must use cryptographically
  secure randomness
\item
  \textbf{Statistical analysis}: Uniform distribution is a simple
  baseline; stronger timing strategies (e.g., exponential/Poisson as in
  Loopix {[}01{]}) provide better resistance to pattern inference
\item
  \textbf{Queue bounds and DoS}: The mixer MUST use a bounded buffer
  with backpressure. Implementations MUST define behavior when full
  (e.g., drop-tail oldest/newest, randomized drop, or reject upstream
  sends) and expose metrics/alerts to prevent memory exhaustion attacks.
\end{itemize}

\subsubsection{5.3. Monitoring and
Metrics}\label{53-monitoring-and-metrics}

The mixer should track:

\begin{itemize}
\tightlist
\item
  Current queue size
\item
  Average packet delay (over configurable window)
\end{itemize}

These metrics aid in:

\begin{itemize}
\tightlist
\item
  Performance tuning
\item
  Detecting abnormal traffic patterns
\item
  Capacity planning
\end{itemize}

\subsection{6. Security Considerations}\label{6-security-considerations}

\subsubsection{6.1. Threat Model}\label{61-threat-model}

The mixer defends against:

\begin{itemize}
\tightlist
\item
  \textbf{Timing correlation attacks}: Randomized delays make linking
  input/output packets by timing significantly harder
\item
  \textbf{Statistical traffic analysis}: Random delays reduce pattern
  predictability but do not eliminate all analysis
\item
  \textbf{Queue manipulation}: Authenticated packet handling prevents
  injection attacks
\end{itemize}

\subsubsection{6.2. Limitations}\label{62-limitations}

The mixer does not protect against:

\begin{itemize}
\item
  low volume spread traffic that does not produce sufficient amount of
  messages to be mixed within the delay window
\item
  \textbf{Global passive adversaries}: With unlimited observation
  capability
\item
  \textbf{Active attacks}: Packet dropping or delaying by malicious
  nodes
\item
  \textbf{Side channels}: CPU, memory, or network-level information
  leaks
\end{itemize}

\subsection{7. Drawbacks}\label{7-drawbacks}

\begin{itemize}
\tightlist
\item
  \textbf{Increased latency}: Every packet experiences additional delay
\item
  \textbf{Memory usage}: Buffering packets requires memory proportional
  to traffic volume and queue size
\item
  \textbf{Complexity}: Adds another component to the protocol stack
  which even makes node-local debugging harder
\item
  \textbf{Simplistic nature}: The mixing does not account for the total
  count of elements in the buffer, with increasing amounts of messages
  in the mixer the generated delay can decrease without sacrificing the
  mixing properties.
\end{itemize}

\subsection{8. Alternatives}\label{8-alternatives}

Alternative mixing strategies considered:

\begin{itemize}
\tightlist
\item
  \textbf{Batch mixing}: Release packets in fixed-size batches (higher
  latency)
\item
  \textbf{Threshold mixing}: Release when buffer reaches certain size
  (variable latency)
\item
  \textbf{Stop-and-go mixing}: Fixed delays at each hop (predictable
  patterns)
\item
  \textbf{Poisson mixing}: As implemented in Loopix {[}01{]}, uses
  Poisson-distributed delays that make real traffic harder to
  distinguish from cover traffic. This can provide stronger anonymity
  properties but requires careful parameter tuning and integration with
  cover traffic.
\end{itemize}

The current continuous mixing approach with uniform distribution is a
simple baseline that balances latency and anonymity while being easier
to implement and analyze.

\subsection{9. Unresolved Questions}\label{9-unresolved-questions}

\begin{itemize}
\tightlist
\item
  Optimal delay parameters for different network conditions
\item
  Adaptive delay strategies based on traffic patterns
\item
  Integration with node-local cover traffic generation
\item
  Memory usage limits and robust overflow handling strategies
\end{itemize}

\subsection{10. Future Work}\label{10-future-work}

\begin{itemize}
\tightlist
\item
  \textbf{Poisson Mixing Implementation}: Implement Poisson mixing
  (exponentially distributed per-packet delays derived from a Poisson
  process) as described in Loopix {[}01{]} to provide stronger anonymity
  properties when combined with cover traffic
\item
  Performance optimizations for hardware acceleration
\end{itemize}

\subsection{11. References}\label{11-references}

{[}01{]} Piotrowska, A. M., Hayes, J., Elahi, T., Meiser, S., \&
Danezis, G. (2017). \href{https://arxiv.org/pdf/1703.00536.pdf}{The
Loopix Anonymity System}. \emph{26th USENIX Security Symposium},
1199-1216.
