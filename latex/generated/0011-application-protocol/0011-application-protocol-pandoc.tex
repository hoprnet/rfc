\section{RFC-0011 Application Layer
protocol}\label{rfc-0011-application-layer-protocol}

\begin{itemize}
\tightlist
\item
  \textbf{RFC Number:} 0011
\item
  \textbf{Title:} Application Layer protocol
\item
  \textbf{Status:} Draft
\item
  \textbf{Author(s):} Lukas Pohanka (@NumberFour8)
\item
  \textbf{Created:} 2025-08-22
\item
  \textbf{Updated:} 2025-08-22
\item
  \textbf{Version:} v0.1.0 (Draft)
\item
  \textbf{Supersedes:} N/A
\item
  \textbf{Related Links:}
  \href{../RFC-0002-mixnet-keywords/0002-mixnet-keywords.md}{RFC-0002},
  \href{../RFC-0004-hopr-packet-protocol/0004-hopr-packet-protocol.md}{RFC-0004},
  \href{../RFC-0008-session-protocol/0008-session-protocol.md}{RFC-0008},
  \href{../RFC-0009-session-start-protocol/0009-session-start-protocol.md}{RFC-0009},
  \href{../RFC-0010-automatic-path-discovery/0010-automatic-path-discovery.md}{RFC-0010}
\end{itemize}

\subsection{1. Abstract}\label{1-abstract}

This RFC describes the Application layer protocol used in the HOPR
project. Typically, this protocol is used in between the HOPR Packet
protocol
\href{../RFC-0004-hopr-packet-protocol/0004-hopr-packet-protocol.md}{RFC-0004}
and some higher-level protocol, such as the Session protocol
\href{../RFC-0008-session-protocol/0008-session-protocol.md}{RFC-0008}
or Start protocol
\href{../RFC-0009-session-start-protocol/0009-session-start-protocol.md}{RFC-0009}.
The goal of this protocol is for a HOPR node to make distinction between
different protocol running on top of the HOPR packet protocol.

It can be seen similar to how standard TCP or UDP protocols
distinguishes between applications using port numbers.

\subsection{2. Motivation}\label{2-motivation}

The HOPR network supports multiple upper layer protocols that serve
different purposes. Without a standardized method to distinguish between
these protocols, nodes would be unable to properly route and handle
packets intended for specific applications. The Application layer
protocol solves this by providing a lightweight tagging mechanism
similar to port numbers in TCP/UDP.

\subsection{3. Terminology}\label{3-terminology}

The key words "MUST", "MUST NOT", "REQUIRED", "SHALL", "SHALL NOT",
"SHOULD", "SHOULD NOT", "RECOMMENDED", "MAY", and "OPTIONAL" in this
document are to be interpreted as described in
\href{https://datatracker.ietf.org/doc/html/rfc2119}{IETF RFC 2119}
when, and only when, they appear in all capitals, as shown here.

Terms defined in
\href{../RFC-0002-mixnet-keywords/0002-mixnet-keywords.md}{RFC-0002}
might be also used.

\subsection{4. Introduction}\label{4-introduction}

The HOPR network can host multitude of upper layer protocols, that serve
different purposes. Some of those are described in other RFCs, such as
\href{../RFC-0008-session-protocol/0008-session-protocol.md}{RFC-0008},
\href{../RFC-0009-session-start-protocol/0009-session-start-protocol.md}{RFC-0009}
or
\href{../RFC-0010-automatic-path-discovery/0010-automatic-path-discovery.md}{RFC-0010}.
The Application layer protocol described in this RFC creates a thin
layer between the HOPR Packet protocol from
\href{../RFC-0004-hopr-packet-protocol/0004-hopr-packet-protocol.md}{RFC-0004}
and these upper layer protocols.

The Application layer protocol primarily serves two purposes:

\begin{enumerate}
\def\labelenumi{\arabic{enumi}.}
\tightlist
\item
  node should be able to distinguish between upper protocols and
  dispatch their packets the respective protocol interpreters
\item
  create an inter-protocol communication link for signals between the
  HOPR Packet protocol and the upper layer protocol
\end{enumerate}

\subsection{5. Specification}\label{5-specification}

The Application layer protocol acts as a wrapper to arbitrary upper
layer \texttt{data} and adds a \texttt{Tag} that determineds the type of
the upper-layer protocol:

\begin{verbatim}
ApplicationData {
    tag: Tag,
    data: [u8; <length>]
    flags: u8
}
\end{verbatim}

The \texttt{Tag} itself MUST be represented by 64 bits and the 3 upper
most significant bits MUST be always set to 0 in the current version.
The remaining 61 bits represent a unique identifier of the upper layer
protocol.

The \texttt{Tag} range SHOULD be split as follows:

\begin{itemize}
\tightlist
\item
  \texttt{0x0000000000000000} identifies the Probing protocol (see
  \href{../RFC-0010-automatic-path-discovery/0010-automatic-path-discovery.md}{RFC-0010}).
\item
  \texttt{0x0000000000000001} identifies the Start protocol (see
  \href{../RFC-0009-session-start-protocol/0009-session-start-protocol.md}{RFC-0009}).
\item
  \texttt{0x0000000000000002} - \texttt{0x000000000000000d} identifies
  range for user protocols
\item
  \texttt{0x000000000000000e} identifies a catch-all for unknown
  protocols
\item
  \texttt{0x000000000000000f} - \texttt{0x1fffffffffffffff} identifes a
  space reserved for the Session protocol (see
  \href{../RFC-0008-session-protocol/0008-session-protocol.md}{RFC-0008}).
\end{itemize}

\subsubsection{5.1 Wire format encoding}\label{51-wire-format-encoding}

The individual fields of \texttt{ApplicationData} MUST be encoded in the
following order:

\begin{enumerate}
\def\labelenumi{\arabic{enumi}.}
\tightlist
\item
  \texttt{tag}: unsigned 8 bytes, big-endian order, the 3 most
  significant bits MUST be cleared
\item
  \texttt{data}: opaque bytes, the length MUST be most the size of the
  HOPR protocol packet, the upper layer protocol SHALL be responsible
  for the framing
\item
  \texttt{field}: MUST NOT be serialized, it is a transient,
  implementation-local, per-packet field
\end{enumerate}

The upper layer protocol MAY use the 4 most significant bits in
\texttt{flags} to pass arbitrary signaling to the HOPR Packet protocol.
Conversely, the HOPR packet protocol MAY use the 4 least significant
bits in \texttt{flags} to pass arbibrary signalling to the upper-layer
protocol.

The interpretation of \texttt{flags} is entirely implementation specific
and MAY be ignored by either sides.

\subsection{6. Appendix 1}\label{6-appendix-1}

\subsubsection{HOPR packet protocol signals in the current
implementation}\label{hopr-packet-protocol-signals-in-the-current-implementation}

The version 1 of the HOPR packet protocol (as in
\href{../RFC-0004-hopr-packet-protocol/0004-hopr-packet-protocol.md}{RFC-0004})
MAY currently pass the following signals to the upper-layer protocol:

\begin{enumerate}
\def\labelenumi{\arabic{enumi}.}
\tightlist
\item
  \texttt{0x01}: SURB distress signal. Indicates that the level of SURBs
  at the counterparty has gone below a certain pre-defined threshold.
\item
  \texttt{0x03}: Out of SURBs signal. Indicates that the received packet
  has used the last SURB available to the Sender.
\end{enumerate}

It is OPTIONAL for any upper-layer protocol to react to these signals if
they are passed to them.

\subsection{7. References}\label{7-references}

None.
